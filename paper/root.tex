\documentclass[letterpaper, 10 pt, conference]{ieeeconf}
%\documentclass[a4paper, 10pt, conference]{ieeeconf}

\IEEEoverridecommandlockouts
\overrideIEEEmargins

% The following packages can be found on http:\\www.ctan.org
\usepackage{graphics} % for pdf, bitmapped graphics files
%\usepackage{epsfig} % for postscript graphics files
%\usepackage{mathptmx} % assumes new font selection scheme installed
%\usepackage{times} % assumes new font selection scheme installed
\usepackage{amsmath} % assumes amsmath package installed
\usepackage{amssymb}  % assumes amsmath package installed

\usepackage{url}
\usepackage{ae} % Nice Slovak symbols
\usepackage{listings,lstautogobble} % Nice MATLAB source code
\usepackage{matlab-prettifier}

\title{\LARGE \bf
Software package for industrial-oriented MPC design and tuning
}

\author{Juraj Holaza, Lenka Gal\v{c}\'{i}kov\'{a}, Juraj Oravec, Michal Kvasnica
\thanks{Authors are with Faculty of Chemical and Food Technology,
		Slovak University of Technology in Bratislava, Bratislava, SK-812 37, Slovakia
        {\tt\small albert.author@papercept.net}}
}

\begin{document}



\maketitle
\thispagestyle{empty}
\pagestyle{empty}

\begin{abstract}

[ TOOD: Think of the better title. ]

[ TOOD: Write an Abstract ]

\end{abstract}

\section{Introduction}
\label{sec:introduction}

[ TOOD: Write an Introduction ]


\section{Theoretical backgrounds}
\label{sec:tube_mpc}

[ TOOD: Write the theoretical backgrounds on Tube MPC. ]

Consider an uncertain linear time-invariant (ULTI) system in the form:
\begin{eqnarray}
	\label{eq:ulti_system}
	x(t+T_{\mathrm{s}}) = A x(t) + B u(t) + E d(t), % \quad x(0) = x_{0},
\end{eqnarray}
where $t$ stands for the time instant in the discrete-time domain determined by the given sampling time $T_{\mathrm{s}}$. $A \in \mathbb{R}^{n_{\mathrm{n}_{x}} \times n_{\mathrm{n}_{x}}}$ is system matrix, $B \in \mathbb{R}^{n_{\mathrm{n}_{x}} \times n_{\mathrm{n}_{u}}}$ is input matrix, such that the matrix pair $(A,B)$ is stabilizable. $E \in \mathbb{R}^{n_{\mathrm{n}_{x}} \times n_{\mathrm{n}_{w}}}$ is disturbance matrix, $x \in \mathbb{R}^{n_{\mathrm{x}}}$ is the vector of the system states, $u \in \mathbb{R}^{n_{\mathrm{u}}}$ is control action, $d \in \mathbb{D} \subset \mathbb{R}^{n_{\mathrm{x}}}$ is bounded additive disturbance such that $\mathbb{D}$ is compact set containing the origin. 

For the sake of notation, in ULTI system~\eqref{eq:ulti_system} holds:
\begin{eqnarray}
	\label{eq:disturbance_set}
	w = E \, d, ~ w \in \mathbb{W}, ~ \mathbb{W} = \left\{ w \in \mathbb{R}^{n_{\mathrm{x}}} : \Vert w \Vert_{\infty} \leq w_{\max} \right\}
\end{eqnarray}
for given upper bound value $w_{\max} = \Vert E \, d \Vert_{\infty}$, $\forall d \in \mathbb{D}$ such that $\mathbb{W} \supseteq E \, \mathbb{D}$ is the minimum volume hyper-box satisfying $\Vert w \Vert_{\infty} = w_{\max}$.

Moreover, the ULTI system in~\eqref{eq:ulti_system} is subject to state and input constraints
\begin{eqnarray}
	\label{eq:constraints_x_u}
	x(t) \in \mathbb{X}, \quad u(t) \in \mathbb{U}, \quad \forall t \geq 0,
\end{eqnarray}
where $\mathbb{X} \in \mathbb{R}^{n_{\mathrm{x}}}$ and $\mathbb{U} \in \mathbb{R}^{n_{\mathrm{u}}}$ are polytopes, i.e., compact sets, containing origin in their strict interiors. 

The ``tube'' introduced into the MPC design is evaluated by the convex set $\mathbb{T} \subset \mathbb{R}^{n_{\mathrm{x}}}$ and represents the origin for the perturbed system, see~\cite{MS05}. 
%
By plugging the LQ-optimal controller into~\eqref{eq:ulti_system} and updating the additive disturbances per~\eqref{eq:disturbance_set}, we obtain an autonomous discrete-time uncertain system 
\begin{equation}
	\label{eq:autosys}
	x(t+T_{\mathrm{s}}) = (A + BK)x(t) + w(t).
\end{equation}
Subsequently, if $\mathbb{T}$ is a robust positive invariant (RPI) set for~\eqref{eq:autosys}, then we have that following statement holds 
\begin{eqnarray}
	\label{eq:tube}
	\left( A + B K \right) \mathbb{T} \oplus \mathbb{W} \subseteq \mathbb{T},
\end{eqnarray}
where $\oplus$ denotes the Minkowski sum.

Obviously, $\mathbb{T}$ in \eqref{eq:tube} is constructed as the minimal RPI set to minimize the conservativeness of Tube MPC design by:
\begin{eqnarray}
	\label{eq:tube_design}
	\mathbb{T} = \sum_{i=0}^{\infty} \left( A + B K \right)^{i} \mathbb{W},
\end{eqnarray}
where $\Sigma$ represents a set addition. However, if the $K$ is not a dead-beat controller, then the minimal RPI set $\mathbb{T}$ does not necessarily lead to the polytope, see~\cite{MS05}. Therefore, $\mathbb{T}$ is designed as the outer approximation of the minimal RPI set.  Algorithm constructing such invariant approximations of the minimal robust positively invariant set $\mathbb{T}$ is proposed in detail in~\cite{RK05}.

The conventional (rigid) Tube MPC design problem has the form:
\begin{subequations}
	\label{eq:tmpc}
	\begin{eqnarray}
		\label{eq:tmpc_cost}
		\min_{\hat{u}_{0},\ldots,\hat{u}_{N-1}, \hat{x}_{0},\ldots,\hat{x}_{N} } \!\!\!\!\!\!\!\!\!\!\! &\,& \Vert \hat{x}_{N} \Vert_{P}^{2} + \sum_{k=0}^{N-1} \left( \Vert \hat{x}_{k} \Vert_{Q}^{2} + \Vert \hat{u}_{k} \Vert_{R}^{2} \right) \qquad \\
		\label{eq:tmpc_rpi}
		\mathrm{s.t.\!:} &\,& x(t) - \hat{x}_{0} \in \mathbb{T} , \\
		\label{eq:tmpc_model}
		&\,&  \hat{x}_{k+1} = A \hat{x}_{k} + B \hat{u}_{k} , \\
		\label{eq:tmpc_constraints_state}
		&\,& \hat{x}_{k} \in \mathbb{X} \ominus \mathbb{T} , \\
		\label{eq:tmpc_constraints_terminal}
		&\,& \hat{x}_{N} \in \mathbb{X}_{\mathrm{N}}, \\
		\label{eq:tmpc_constraints_input}
		&\,& \hat{u}_{k} \in \mathbb{U} \ominus K \, \mathbb{T} , \\
		\label{eq:tmpc_constraints_input_delta_k}
		&\,& \Delta \hat{u}_{l} \in \mathbb{U}_{\Delta} \ominus K \, \mathbb{T} , \\
		\label{eq:tmpc_constraints_input_delta_0}
		&\,& \hat{u}_{k} + K ( x(t) - \hat{x}_{0} ) - u(t_{-}) \in \mathbb{U}_{\Delta} , \quad
	\end{eqnarray}
\end{subequations}
where $\forall k = 0, \dots, N-1$,  $\forall l = 1, \dots, N-1$, $N$ is prediction horizon.
The decision variables\footnote{Obviously, dense formulation of~\eqref{eq:tmpc} omits $\hat{x}_{k}$ as the decision variables, except for $\hat{x}_{0}$.} $\hat{u}_{k}$, $\hat{x}_{k}$, are optimized subject to the nominal system behaviour in~\eqref{eq:tmpc_model}, i.e., an idealized system without the impact of any uncertain parameters. The state and input constraints in~\eqref{eq:constraints_x_u} are respectively adopted in~\eqref{eq:tmpc_constraints_state}, \eqref{eq:tmpc_constraints_input} to respect the RPI set $\mathbb{T}$ in~\eqref{eq:tube} assuming: $(\mathbb{X} \ominus \mathbb{T})$, $(\mathbb{U} \ominus K \, \mathbb{T})$ are non-empty sets, convex by definition. Analogous, the initial condition in~\eqref{eq:tmpc_rpi} takes into account the RPI set $\mathbb{T}$ keeping the perturbed system state vector $x(t)$ close to its nominal counterpart $\hat{x}_{0}$. 
The terminal constraint in~\eqref{eq:tmpc_constraints_terminal} has the conventional form.
The quadratic cost function in~\eqref{eq:tmpc_cost} is minimized considering the $Q$, $R$, $P$. Note, $\Vert \hat{x}_{N} \Vert_{P}^{2}$ denotes simplified notation for the weighted two norm: $\hat{x}_{N}^{\top} P \hat{x}_{N}$, and analogous hold for the remaining terms. 
The corresponding stability and recursive feasibility proofs of~\eqref{eq:tmpc} are documented in~\cite{MS05}.

The control action $u(t)$, which is applied to the controlled plant in~\eqref{eq:ulti_system}, is determined by the control law $\kappa : \mathbb{X}_{\mathrm{F}} \rightarrow \mathbb{U}$
\begin{eqnarray}
	\label{eq:tmpc_control_law}
	\kappa(x(t)) = \hat{u}_{0}^{\star} + K \left( x(t) - \hat{x}_{0}^{\star} \right),
\end{eqnarray}
where $\star$ denotes the solution of the optimization problem in~\eqref{eq:tmpc} and $\mathbb{X}_{\mathrm{F}} \subseteq \mathbb{R}^{n_{\mathrm{x}}}$ is the corresponding domain, i.e., the feasibility set of the optimized initial conditions $\hat{x}_{0}$ of~\eqref{eq:tmpc}. 
Tube MPC is implemented in receding horizon fashion, i.e., just the first control action $u(t) = \kappa(x(t))$ is applied to the plant and the optimization problem in~\eqref{eq:tmpc} is re-computed in each control step. 

\section{The software package}
\label{sec:code}

[ TOOD: Describe a software package. ]

\subsection{MPT+}
\label{sec:code_mptplus}

[ TOOD: Describe an MPT+ package. ]

\subsection{Advanced methods for explicit Tube MPC controllers}
\label{sec:advanced_method}

% [ TODO: Write about the advanced methods of the MPT toolbox available for Tube MPC. ]
This section lists the current compatibility\footnote{The package \texttt{MPTplus} is still under development and further extensions are expected, soon.} of \texttt{MPTplus} with \texttt{MPT}. 
%
Since the implicit MPC policy \verb|TMPC| is currently restricted only to evaluation via \verb|TMPC.optimizer(x)|, for any feasible state vector $x\in\mathbb{X}_{\mathrm{F}}$, we devote this section to its explicit counterpart \verb|ETMPC|.

Assume that from the previous Section~\ref{sec:construction} we have defined \verb|model|, i.e., object that contains prediction model~\eqref{eq:example_ulti_system} as in~\eqref{eq:tmpc_model}, constraints~\eqref{eq:constraints_x_u} transformed into~\eqref{eq:tmpc_constraints_state}-\eqref{eq:tmpc_constraints_input}, penalty matrices, the appertain LQR terminal penalty and terminal set as in~\eqref{eq:tmpc_cost} and~\eqref{eq:tmpc_constraints_terminal}, respectively. 
%
Using the \texttt{MPTplus}, we may simply construct both, non-explicit and explicit Tube MPC controllers\footnote{The current version of \texttt{MPTplus} always displays for explicit Tube MPC controller confusing information on the prediction horizon $N=1$, although the controller is constructed correctly for any value of $N$.}. To design explicit MPC policy that returns the control action as in~\eqref{eq:tmpc_control_law}, i.e., that is fed directly to the controlled system in~\eqref{eq:ulti_system}, we can evoke:
\begin{lstlisting}[style=Matlab-editor]
ops = {'eMPC',1}
[~,ETMPC] = TMPCController(model,N,ops)
\end{lstlisting}
what constructs explicit controller \verb|ETMPC| that is defined over $n_\text{r} = 565$ critical regions in $n_\text{x} = 2$ dimensional space.

To graphically analyze our explicit controller, we can type:
\begin{lstlisting}[style=Matlab-editor]
	figure; ETMPC.partition.plot()
	figure; ETMPC.feedback.fplot()
	figure; ETMPC.cost.fplot()
\end{lstlisting}
to plot the polytopic partition~\eqref{eq:tmpc_partition}, the PWA feedback law~\eqref{eq:tmpc_control_law_pwa}, and the PWQ value function as defined in~\eqref{eq:tmpc_cost}, respectively.\footnote{We note that while both, the PWA feedback law and the PWQ value function, are given as solution to~\eqref{eq:tmpc} the solution~\eqref{eq:tmpc_control_law_pwa_particular} was posterior transformed into~\eqref{eq:tmpc_control_law_pwa}, however, the value function preserves its form of~\eqref{eq:tmpc_cost}.}

To perform a closed-loop simulation of a computed explicit Tube MPC controller one can define an object:
\begin{lstlisting}[style=Matlab-editor]
	loop = ClosedLoop(ETMPC,model)
\end{lstlisting}
where \verb|model| can contain even a modified version of~\eqref{eq:example_ulti_system}, i.e., the controlled system can be different from the prediction model used to design the explicit policy \verb|ETMPC|. Subsequently, we can execute the closed-loop simulation by:
\begin{lstlisting}[style=Matlab-editor]
	Nsim = 10
	data = loop.simulate(x0, Nsim)
\end{lstlisting}	
with \verb|x0| denoting the initial condition and \verb|Nsim| number of the closed-loop simulation steps. The generated \verb|data| structure contains all important information, e.g., the closed-loop profiles of control inputs \verb|data.U|, applied disturbances \verb|data.D|, states \verb|data.X|, and the cost of~\eqref{eq:tmpc_cost} at each time step, to name a few.
Alternatively, one can create a customized loop, where the control input of \verb|ETMPC| can be obtained by typing: 
\begin{lstlisting}[style=Matlab-editor]
	u = ETMPC.evaluate(x)
\end{lstlisting}
for any given feasible state vector $x\in\mathbb{X}_{\mathrm{F}}$.
%
Finally, if only pure visual analysis of a closed-loop simulation is required then we encourage users to use
\begin{lstlisting}[style=Matlab-editor]
	ETMPC.clicksim()
\end{lstlisting}
where initial state conditions are defined by the mouse.
%
%	Finally, for more rigorous verification of closed-loop stability we encourage a user to create a Lyapunov function $V$ that provides $V(x(k))>0$, $V(0)=0$, $V(x(k+1)) \leq \gamma V(x(k))$ for all $x(k) \in \mathbb{X}_{\mathrm{F}}$, some $\gamma \in [0,1)$, and all $k \geq 0$. This can be done easily by 
%	\begin{lstlisting}[style=Matlab-editor]
	%		lyap = loop.toSystem.lyapunov(type)
	%	\end{lstlisting}

Another interesting group of \texttt{MPT} features is the module \texttt{LowCom}~\cite{KH15} that provides various complexity reduction techniques of explicit MPC policies. In this paper, we list techniques that do not induce suboptimality\footnote{Currently, complexity reduction schemes that introduce suboptimality are not compatible with the Tube MPC.}.

The Optimal Region Merging (ORM) method aims to join critical regions with the same closed-loop feedback policy and whose union is a convex set. This method can be called: 
\begin{lstlisting}[style=Matlab-editor]
ETMPCsim = ETMPC.simplify('orm')
\end{lstlisting}
where \verb|ETMPCsim| is the constructed simplified controller. In our setup, this technique found $158$ critical regions that were merged. Hence, we have reduced the complexity by $28\%$ as the number of critical regions dropped from $565$ to $407$. 

%
%	The clipped-based method targets to achieve complexity reduction by removing all critical regions with saturated control law. Subsequently, to cover the entire original polytopic partition $\mathbb{X}_{\mathrm{F}}$, critical regions with unsaturated control law are extended and a clipping filter is introduced. This method can be called by 
%	\begin{lstlisting}[style=Matlab-editor]
	%	simpleETMPC = ETMPC.simplify('clipping')
	%	\end{lstlisting}
%	where \verb|simpleETMPC| denotes the resulting simplified controller. With our setup, 
%
The separation-based method targets to reduce the complexity of explicit controllers by removing all critical regions that lead to the saturated control laws, i.e., $u(t)$ that lay on the boundary of $\mathbb{U}$. Once this is done, a separation filter is constructed to identify optimal control actions above the removed critical regions. This simplification method is called:
\begin{lstlisting}[style=Matlab-editor]
ETMPCsim = ETMPC.simplify('separation')
\end{lstlisting}
where \verb|ETMPCsim| denotes the resulting simplified controller. With the considered control setup, the separation-based controller has removed all $120$ regions with saturated feedback law which represents the reduction of $21\%$ w.r.t. its complex counterpart \verb|ETMPC|.

Needlessly to say, there are many advanced methods that can be used on top of the \verb|ETMPC| object that, due to paper limitations, can not be listed here. All interested readers are hence referred to the dedicated \texttt{MPT} website: \\ \texttt{\url{https://www.mpt3.org}}.

\subsection{LowCom}
\label{sec:code_mpt3lowcom}

[ TOOD: Recall an LowCom package. ]

\section{Case Study}
\label{sec:case_study}

[ TOOD: Write a laboratory case study on Flexy control. ]

\subsection{Control setup}
\label{sec:control_setup}

[ TOOD: Write a detailed control setup. ]

\subsection{Results and discussion}
\label{sec:results}

[ TOOD: Discuss the results. ]

\subsection{Figures and Tables}

\begin{table}[h]
\caption{An Example of a Table}
\label{table_example}
\begin{center}
\begin{tabular}{|c||c|}
\hline
One & Two\\
\hline
Three & Four\\
\hline
\end{tabular}
\end{center}
\end{table}


\begin{figure}[thpb]
      \centering
      \framebox{\parbox{3in}{We suggest that you use a text box to insert a graphic (which is ideally a 300 dpi TIFF or EPS file, with all fonts embedded) because, in an document, this method is somewhat more stable than directly inserting a picture.
}}
      %\includegraphics[scale=1.0]{figurefile}
      \caption{Inductance of oscillation winding on amorphous
       magnetic core versus DC bias magnetic field}
      \label{figurelabel}
   \end{figure}
   
\section{Conclusions}
\label{sec:conclusions}

[ TOOD: Summarize the main conclusions. ]

\addtolength{\textheight}{-12cm}

\section*{Acknowledgemnts}

[ TODO: Double-check Acks. ]

The authors gratefully acknowledge the contribution of the Scientific Grant Agency of the Slovak Republic under the grants 1/0545/20, 1/0297/22, and the Slovak Research and Development Agency under the project APVV-20-0261. This research is funded by the European Union?s Horizon Europe under grant no. 101079342 (Fostering Opportunities Towards Slovak Excellence in Advanced Control for Smart Industries).


\bibliographystyle{IEEEtran} % use IEEEtran.bst style
% \small
\bibliography{references}


\end{document}
